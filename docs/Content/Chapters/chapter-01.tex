%!TEX root = ../../thesis.tex
\chapter{Introduction}

A brain tumor is a growth of cells in the brain or near it. Brain tumors can happen
in the brain tissue. Brain tumors also can happen near the brain tissue. Nearby
locations include nerves, the pituitary gland, the pineal gland, and the membranes
that cover the surface of the brain.
Your skull, which encloses your brain, is very rigid. Any growth inside such a
restricted space can cause problems. Brain tumors can be of two types i.e.
cancerous (malignant) or non-cancerous (benign).
Brain cancer is a life-threatening disease, and it affects all the diagnosed people
severely. Precise brain tumor classification helps to diagnose brain cancer early,
which increases the survival rate of brain cancer patients, but it is quite hard to
detect early. It is difficult to evaluate the magnetic resonance imaging images
manually. Therefore, there is a need for optimized and fast digital methods for
tumor diagnosis with better accuracy.
In this work, we are using a deep learning model based on convolutional neural
networks to identify and classify brain cancer and its type from MRI images of
patients using publicly available datasets in Kaggle \footnote{\url{https://www.kaggle.com}}.

For brain stroke, it is a brain attack and it is a sudden interruption of continuous
blood flow to the brain and a medical emergency. A stroke occurs when a blood
vessel in the brain becomes blocked or narrowed, or when a blood vessel bursts
and spills blood into the brain.
Stroke is a major contributor to death and disability worldwide. Before modern
therapy, post-stroke mortality was approximately 10% in the acute period, with
nearly one-half of the patients develop moderate-to-severe disability.
The most fundamental aspect of acute stroke management is “time is brain”. In acute
ischemic stroke, the primary therapeutic goal of reperfusion therapy, including
intravenous recombinant tissue plasminogen activator (IV TPA) and/or
endovascular thrombectomy is the rapid restoration of cerebral blood flow to the
salvageable ischemic brain tissue at risk for cerebral infarction. Several landmarks
endovascular thrombectomy trials were found to be of benefit in selecting patients
with acute stroke caused by occlusion of the proximal anterior circulation, which
has led to a paradigm shift in the management of acute ischemic strokes.

In this modern era of acute stroke care, more patients will survive with varying degrees of
disability post-stroke. A comprehensive stroke rehabilitation program is critical to
optimize post-stroke outcomes. Understanding the natural history of stroke
recovery, and adapting a multidisciplinary approach, will lead to improved chances
for successful rehabilitation.

Here we are using a CNN model for deep learning multi-image classification which classifies input images into tumor, no tumor, stroke, or no stroke.

% \begin{figure}
%     \centering
%     \includegraphics[width=0.25\textwidth]{Img/Chap-01/harry3.jpg}
%     \caption[Horcrux Hufflepuff’s Cup]{Horcrux Hufflepuff’s Cup \citep{horcrux_fandom}.}
%     \label{fig:hufflepuff_cup}
% \end{figure}
% Through rigorous research and practical experimentation, this work explores the foundational principles of Horcruxes, uncovering the detailed mechanisms and magical properties that make them nearly indestructible. The study examines the historical context of Horcruxes, analyzing ancient texts and previous research to build a solid theoretical framework. Additionally, it addresses the profound psychological and metaphysical consequences of creating Horcruxes, offering new insights into the ethical and magical ramifications.

% By expanding the boundaries of what is known about immortality, this thesis aims to contribute significantly to the field of \ac{DM}. 